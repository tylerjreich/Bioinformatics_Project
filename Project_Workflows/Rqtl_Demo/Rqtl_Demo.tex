% Options for packages loaded elsewhere
\PassOptionsToPackage{unicode}{hyperref}
\PassOptionsToPackage{hyphens}{url}
\documentclass[
]{article}
\usepackage{xcolor}
\usepackage[margin=1in]{geometry}
\usepackage{amsmath,amssymb}
\setcounter{secnumdepth}{-\maxdimen} % remove section numbering
\usepackage{iftex}
\ifPDFTeX
  \usepackage[T1]{fontenc}
  \usepackage[utf8]{inputenc}
  \usepackage{textcomp} % provide euro and other symbols
\else % if luatex or xetex
  \usepackage{unicode-math} % this also loads fontspec
  \defaultfontfeatures{Scale=MatchLowercase}
  \defaultfontfeatures[\rmfamily]{Ligatures=TeX,Scale=1}
\fi
\usepackage{lmodern}
\ifPDFTeX\else
  % xetex/luatex font selection
\fi
% Use upquote if available, for straight quotes in verbatim environments
\IfFileExists{upquote.sty}{\usepackage{upquote}}{}
\IfFileExists{microtype.sty}{% use microtype if available
  \usepackage[]{microtype}
  \UseMicrotypeSet[protrusion]{basicmath} % disable protrusion for tt fonts
}{}
\makeatletter
\@ifundefined{KOMAClassName}{% if non-KOMA class
  \IfFileExists{parskip.sty}{%
    \usepackage{parskip}
  }{% else
    \setlength{\parindent}{0pt}
    \setlength{\parskip}{6pt plus 2pt minus 1pt}}
}{% if KOMA class
  \KOMAoptions{parskip=half}}
\makeatother
\usepackage{color}
\usepackage{fancyvrb}
\newcommand{\VerbBar}{|}
\newcommand{\VERB}{\Verb[commandchars=\\\{\}]}
\DefineVerbatimEnvironment{Highlighting}{Verbatim}{commandchars=\\\{\}}
% Add ',fontsize=\small' for more characters per line
\usepackage{framed}
\definecolor{shadecolor}{RGB}{248,248,248}
\newenvironment{Shaded}{\begin{snugshade}}{\end{snugshade}}
\newcommand{\AlertTok}[1]{\textcolor[rgb]{0.94,0.16,0.16}{#1}}
\newcommand{\AnnotationTok}[1]{\textcolor[rgb]{0.56,0.35,0.01}{\textbf{\textit{#1}}}}
\newcommand{\AttributeTok}[1]{\textcolor[rgb]{0.13,0.29,0.53}{#1}}
\newcommand{\BaseNTok}[1]{\textcolor[rgb]{0.00,0.00,0.81}{#1}}
\newcommand{\BuiltInTok}[1]{#1}
\newcommand{\CharTok}[1]{\textcolor[rgb]{0.31,0.60,0.02}{#1}}
\newcommand{\CommentTok}[1]{\textcolor[rgb]{0.56,0.35,0.01}{\textit{#1}}}
\newcommand{\CommentVarTok}[1]{\textcolor[rgb]{0.56,0.35,0.01}{\textbf{\textit{#1}}}}
\newcommand{\ConstantTok}[1]{\textcolor[rgb]{0.56,0.35,0.01}{#1}}
\newcommand{\ControlFlowTok}[1]{\textcolor[rgb]{0.13,0.29,0.53}{\textbf{#1}}}
\newcommand{\DataTypeTok}[1]{\textcolor[rgb]{0.13,0.29,0.53}{#1}}
\newcommand{\DecValTok}[1]{\textcolor[rgb]{0.00,0.00,0.81}{#1}}
\newcommand{\DocumentationTok}[1]{\textcolor[rgb]{0.56,0.35,0.01}{\textbf{\textit{#1}}}}
\newcommand{\ErrorTok}[1]{\textcolor[rgb]{0.64,0.00,0.00}{\textbf{#1}}}
\newcommand{\ExtensionTok}[1]{#1}
\newcommand{\FloatTok}[1]{\textcolor[rgb]{0.00,0.00,0.81}{#1}}
\newcommand{\FunctionTok}[1]{\textcolor[rgb]{0.13,0.29,0.53}{\textbf{#1}}}
\newcommand{\ImportTok}[1]{#1}
\newcommand{\InformationTok}[1]{\textcolor[rgb]{0.56,0.35,0.01}{\textbf{\textit{#1}}}}
\newcommand{\KeywordTok}[1]{\textcolor[rgb]{0.13,0.29,0.53}{\textbf{#1}}}
\newcommand{\NormalTok}[1]{#1}
\newcommand{\OperatorTok}[1]{\textcolor[rgb]{0.81,0.36,0.00}{\textbf{#1}}}
\newcommand{\OtherTok}[1]{\textcolor[rgb]{0.56,0.35,0.01}{#1}}
\newcommand{\PreprocessorTok}[1]{\textcolor[rgb]{0.56,0.35,0.01}{\textit{#1}}}
\newcommand{\RegionMarkerTok}[1]{#1}
\newcommand{\SpecialCharTok}[1]{\textcolor[rgb]{0.81,0.36,0.00}{\textbf{#1}}}
\newcommand{\SpecialStringTok}[1]{\textcolor[rgb]{0.31,0.60,0.02}{#1}}
\newcommand{\StringTok}[1]{\textcolor[rgb]{0.31,0.60,0.02}{#1}}
\newcommand{\VariableTok}[1]{\textcolor[rgb]{0.00,0.00,0.00}{#1}}
\newcommand{\VerbatimStringTok}[1]{\textcolor[rgb]{0.31,0.60,0.02}{#1}}
\newcommand{\WarningTok}[1]{\textcolor[rgb]{0.56,0.35,0.01}{\textbf{\textit{#1}}}}
\usepackage{graphicx}
\makeatletter
\newsavebox\pandoc@box
\newcommand*\pandocbounded[1]{% scales image to fit in text height/width
  \sbox\pandoc@box{#1}%
  \Gscale@div\@tempa{\textheight}{\dimexpr\ht\pandoc@box+\dp\pandoc@box\relax}%
  \Gscale@div\@tempb{\linewidth}{\wd\pandoc@box}%
  \ifdim\@tempb\p@<\@tempa\p@\let\@tempa\@tempb\fi% select the smaller of both
  \ifdim\@tempa\p@<\p@\scalebox{\@tempa}{\usebox\pandoc@box}%
  \else\usebox{\pandoc@box}%
  \fi%
}
% Set default figure placement to htbp
\def\fps@figure{htbp}
\makeatother
\setlength{\emergencystretch}{3em} % prevent overfull lines
\providecommand{\tightlist}{%
  \setlength{\itemsep}{0pt}\setlength{\parskip}{0pt}}
\usepackage{bookmark}
\IfFileExists{xurl.sty}{\usepackage{xurl}}{} % add URL line breaks if available
\urlstyle{same}
\hypersetup{
  pdftitle={R/qtl Demo},
  pdfauthor={Tyler J Reich},
  hidelinks,
  pdfcreator={LaTeX via pandoc}}

\title{R/qtl Demo}
\author{Tyler J Reich}
\date{2025-12-12}

\begin{document}
\maketitle

\section{Preface on this Project:}\label{preface-on-this-project}

With almost no background in computer science, learning to write code
has been a significant challenge; however, it is one I am eager to
undertake. I am grateful that we are allowed to use generative AI in
this class, as ChatGPT has been an invaluable resource in helping me
develop and understand the R/qtl workflow for QTL mapping. While ChatGPT
assisted me in generating the R code to import, process, and analyze
genotype and phenotype data, I also asked it to explain what each line
of code does and why it is used in the context of QTL analysis. I am
prepared to explain this workflow to the best of my ability in order to
demonstrate my understanding of how and why it works.

\section{Preparing Variant Data for R/qtl
Analysis}\label{preparing-variant-data-for-rqtl-analysis}

This section describes how to convert genomic Variant Call Format
(gVCF/VCF) files from the GATK pipeline into a format compatible with
R/qtl for QTL mapping.

\subsection{Requirements:}\label{requirements}

\begin{enumerate}
\def\labelenumi{\arabic{enumi}.}
\tightlist
\item
  R (\textgreater=4.2 recommended)
\item
  R/qtl package
\item
  bcftools
\item
  vcftools (optional, for filtering or subsetting VCFs) 5.CSV conversion
  script (provided in scripts/)
\end{enumerate}

\begin{center}\rule{0.5\linewidth}{0.5pt}\end{center}

\subsection{1. Obtain VCF Files from the GATK
Pipeline}\label{obtain-vcf-files-from-the-gatk-pipeline}

After low-coverage whole-genome sequencing (lcWGS), single-sample gVCFs
are generated with \textbf{GATK HaplotypeCaller} in reference-confidence
mode. These are then combined and genotyped into a multi-individual VCF
using:

\begin{itemize}
\tightlist
\item
  \texttt{GenomicsDBImport}
\item
  \texttt{GenotypeGVCFs}
\end{itemize}

This final VCF (or one subset for testing) serves as the input for
conversion to R/qtl format.

\begin{center}\rule{0.5\linewidth}{0.5pt}\end{center}

\subsection{2. Convert VCF to CSV for R/qtl and Add Phenotype
Data}\label{convert-vcf-to-csv-for-rqtl-and-add-phenotype-data}

R/qtl expects a single CSV file in which:

\begin{itemize}
\tightlist
\item
  \textbf{Each row = one individual}\\
\item
  \textbf{Columns = individual ID, phenotype values, followed by marker
  genotypes}
\end{itemize}

Because VCFs are marker-major (one row per variant), they must be
transposed into individual-major format.

\subsubsection{Extract genotypes using
bcftools}\label{extract-genotypes-using-bcftools}

Use \texttt{bcftools\ query} to extract marker positions and genotypes:

\section{Extract marker positions and per-individual genotypes from a
VCF
file}\label{extract-marker-positions-and-per-individual-genotypes-from-a-vcf-file}

\begin{verbatim}
(bash) bcftools query -f '%CHROM\t%POS[\t%GT]\n' input.vcf.gz > genotypes_raw.tsv

Notes:
bcftools query -> Calls the query command in bcftools, which extracts specific fields from a VCF using a user-defined format string.
-f '%CHROM\t%POS[\t%GT]\n' -> The format string that tells bcftools exactly what information to print:
%CHROM → chromosome of each variant
%POS → genomic position of the variant
[\t%GT] → for each sample in the VCF, print a tab (\t) followed by the genotype (%GT)
The square brackets indicate “repeat this for each sample.”
\t → literal tab separators
\n → end the line after each variant
\end{verbatim}

Each output row will look like: chrom pos sample1\_GT sample2\_GT
sample3\_GT \ldots{}

input.vcf.gz -\textgreater{} The compressed VCF file produced by GATK
after joint genotyping. genotypes\_raw.tsv -\textgreater{} Redirects the
output into a tab-separated text file named genotypes\_raw.tsv.

This produces: 1. One row per marker 2. One column per individual 3.
Genotypes reported in VCF format (e.g., 0/0, 0/1, 1/1)

Convert to R/qtl format: A provided script (in scripts/) converts this
marker-major file into R/qtl's individual-major CSV format. This script:
1. Transposes the genotype matrix 2. Converts VCF genotypes (0/1) to
R/qtl genotypes (e.g., AB) 3. Adds marker names and chromosome positions
4. Ensures consistent delimiters and header formatting

Add phenotypes: Before loading into R/qtl, phenotype measurements must
be included as columns immediately after the id column.

Once combined, this CSV can be imported directly into R/qtl using:

\begin{verbatim}
cross <- read.cross(format = "csv", file = "filename.csv")
\end{verbatim}

\section{Preface for the R/qtl Analysis
Section}\label{preface-for-the-rqtl-analysis-section}

Before applying the QTL mapping pipeline to my Poecilia F2 dataset, I
validated all downstream analytical steps using a built-in example
dataset provided by R/qtl. The \texttt{listeria} dataset is an
experimentally derived F2 intercross with known genotypes and phenotypes
(Broman et al., 2003), making it an ideal proof-of-concept system. The
following analysis demonstrates that the entire R/qtl workflow from data
import and quality control to genome scans, permutation testing,
confidence interval estimation, and effect-size visualization---runs
successfully on a complete and well-formatted F2 dataset. Once my
Poecilia genotypes (VCFs -\textgreater{} CSV) and phenotypes are
generated, only the input files will need to be replaced, and the same
workflow can be directly applied.

\section{Using Built-In Example Dataset from
R/qtl}\label{using-built-in-example-dataset-from-rqtl}

This section begins by loading the listeria dataset, a built-in F2
intercross included with the R/qtl package. Importing the dataset into
the object cross allows all downstream QTL-mapping functions, such as
quality control, genome scanning, permutation testing, and effect-size
visualization, to be demonstrated on a fully-formatted and validated
example dataset.

\begin{Shaded}
\begin{Highlighting}[]
\FunctionTok{library}\NormalTok{(qtl)}
\end{Highlighting}
\end{Shaded}

\begin{verbatim}
## Warning: package 'qtl' was built under R version 4.3.3
\end{verbatim}

\begin{Shaded}
\begin{Highlighting}[]
\CommentTok{\# Load the listeria F2 intercross dataset}
\FunctionTok{data}\NormalTok{(listeria)}
\NormalTok{cross }\OtherTok{\textless{}{-}}\NormalTok{ listeria}
\end{Highlighting}
\end{Shaded}

\section{Inspect the Cross}\label{inspect-the-cross}

This step inspects the structure of the imported cross object to confirm
that the dataset contains the expected phenotypes, chromosomes, and
genetic markers. The \texttt{summary(cross)} command provides an
overview of the cross type, number of individuals, phenotypic traits,
markers, and missing data, while \texttt{names(cross\$pheno)} and
\texttt{names(cross\$geno)} list the available phenotype columns and
chromosomes, respectively. This ensures the dataset is correctly loaded
and ready for QTL analysis.

\begin{Shaded}
\begin{Highlighting}[]
\FunctionTok{summary}\NormalTok{(cross)}
\end{Highlighting}
\end{Shaded}

\begin{verbatim}
##     F2 intercross
## 
##     No. individuals:    120 
## 
##     No. phenotypes:     2 
##     Percent phenotyped: 96.7 100 
## 
##     No. chromosomes:    20 
##         Autosomes:      1 2 3 4 5 6 7 8 9 10 11 12 13 14 15 16 17 18 19 
##         X chr:          X 
## 
##     Total markers:      133 
##     No. markers:        13 6 6 4 13 13 6 6 7 5 6 6 12 4 8 4 4 4 4 2 
##     Percent genotyped:  88.5 
##     Genotypes (%):    
##           Autosomes:        CC:25.8      CB:48.9      BB:24.4  not BB:0.0  
##                         not CC:0.9  
##        X chromosome:        CC:51.7      CB:48.3
\end{verbatim}

\begin{Shaded}
\begin{Highlighting}[]
\CommentTok{\# Check phenotypes and genotypes}
\FunctionTok{names}\NormalTok{(cross}\SpecialCharTok{$}\NormalTok{pheno)   }\CommentTok{\# phenotypes: "T264", "sex"}
\end{Highlighting}
\end{Shaded}

\begin{verbatim}
## [1] "T264" "sex"
\end{verbatim}

\begin{Shaded}
\begin{Highlighting}[]
\FunctionTok{names}\NormalTok{(cross}\SpecialCharTok{$}\NormalTok{geno)    }\CommentTok{\# chromosomes: 1, 2, ..., 19, X}
\end{Highlighting}
\end{Shaded}

\begin{verbatim}
##  [1] "1"  "2"  "3"  "4"  "5"  "6"  "7"  "8"  "9"  "10" "11" "12" "13" "14" "15"
## [16] "16" "17" "18" "19" "X"
\end{verbatim}

\section{Quality Control Checks}\label{quality-control-checks}

These quality-control checks evaluate data completeness and marker
reliability before performing QTL mapping. The
\texttt{plotMissing(cross)} function visualizes missing genotype data
across individuals and markers, helping identify potential issues such
as poorly performing markers or samples with excessive missingness. The
\texttt{geno.table(cross)} command summarizes genotype frequencies for
each marker, allowing assessment of segregation distortion or unexpected
genotype ratios. Together, these checks ensure the dataset meets quality
standards for accurate downstream QTL analysis.

\begin{Shaded}
\begin{Highlighting}[]
\CommentTok{\# Plot missing data}
\FunctionTok{plotMissing}\NormalTok{(cross)}
\end{Highlighting}
\end{Shaded}

\pandocbounded{\includegraphics[keepaspectratio]{Rqtl_Demo_files/figure-latex/unnamed-chunk-3-1.pdf}}

\begin{Shaded}
\begin{Highlighting}[]
\CommentTok{\# Check segregation ratios}
\FunctionTok{geno.table}\NormalTok{(cross)}
\end{Highlighting}
\end{Shaded}

\begin{verbatim}
##         chr missing CC CB BB not.BB not.CC      P.value
## D10M44    1      19 29 46 26      0      0 0.6125657411
## D1M3      1       3 26 60 31      0      0 0.7771384555
## D1M75     1      24 15 52 29      0      0 0.0930144892
## D1M215    1       3 22 63 32      0      0 0.3009368909
## D1M309    1       2 31 52 35      0      0 0.3805637036
## D1M218    1       2 30 52 36      0      0 0.3212315037
## D1M451    1       0 29 57 34      0      0 0.6988400895
## D1M504    1       0 31 55 34      0      0 0.6116062006
## D1M113    1       0 30 57 33      0      0 0.7985162188
## D1M355    1       0 30 57 33      0      0 0.7985162188
## D1M291    1       5 26 59 30      0      0 0.8367241192
## D1M209    1      12 26 57 25      0      0 0.8386801055
## D1M155    1       5 26 59 30      0      0 0.8367241192
## D2M365    2       7 24 61 28      0      0 0.6065306597
## D2M37     2       1 29 60 30      0      0 0.9874740688
## D2M396    2       3 27 62 28      0      0 0.8041666799
## D2M493    2      30 24 44 22      0      0 0.9355069850
## D2M226    2       9 33 54 24      0      0 0.4628879466
## D2M148    2       0 34 62 24      0      0 0.4065696597
## D3M265    3       0 33 66 21      0      0 0.1652988882
## D3M51     3      33 26 46 15      0      0 0.2155671626
## D3M106    3      30 28 45 17      0      0 0.2606844924
## D3M257    3       2 33 69 16      0      0 0.0158582121
## D3M147    3       5 29 69 17      0      0 0.0286622626
## D3M19     3      30 23 50 17      0      0 0.3845984194
## D4M2      4      30 18 49 23      0      0 0.5308194506
## D4M178    4      44 12 46 18      0      0 0.1155681665
## D4M187    4      30 13 48 29      0      0 0.0476227617
## D4M251    4       0 22 62 36      0      0 0.1826835241
## D5M148    5      11 27 47 35      0      0 0.1980439991
## D5M232    5       7 26 53 34      0      0 0.4569479266
## D5M257    5       0 32 55 33      0      0 0.6537697851
## D5M83     5       0 31 56 33      0      0 0.7408182207
## D5M307    5       4 32 53 31      0      0 0.6442585409
## D5M357    5       0 31 58 31      0      0 0.9355069850
## D5M205    5       4 27 56 33      0      0 0.6843332004
## D5M398    5      57 15 30 18      0      0 0.8071177470
## D5M91     5       0 27 59 34      0      0 0.6537697851
## D5M338    5       0 27 61 32      0      0 0.7985162188
## D5M188    5       1 28 57 34      0      0 0.6652695213
## D5M29     5      58 13 25 24      0      0 0.0444716565
## D5M168    5       0 30 52 38      0      0 0.2018965180
## D6M223    6       1 32 59 28      0      0 0.8705279066
## D6M188    6      30 21 54 15      0      0 0.1108031584
## D6M284    6       0 27 76 17      0      0 0.0060967466
## D6M39     6       0 24 75 21      0      0 0.0218184355
## D6M254    6       0 18 77 25      0      0 0.0053803600
## D6M194    6      30 13 54 23      0      0 0.0544152349
## D6M290    6      53 13 35 19      0      0 0.5463597639
## D6M25     6       4 20 67 29      0      0 0.1230914645
## D6M339    6       0 22 68 30      0      0 0.2018965180
## D6M59_    6      30 16 49 25      0      0 0.2849175167
## D6M201    6       1 23 66 30      0      0 0.3256771455
## D6M15     6       1 22 67 30      0      0 0.2269119375
## D6M294    6       0 19 70 31      0      0 0.0568882383
## D7M246    7      30 22 42 26      0      0 0.6853827910
## D7M145    7       4 28 58 30      0      0 0.9661049965
## D7M62     7       3 28 61 28      0      0 0.8986715993
## D7M126    7      27 29 37 27      0      0 0.1375358089
## D7M105    7       0 30 56 34      0      0 0.6703200460
## D7M259    7      46 17 41 16      0      0 0.6402183775
## D8M94     8       0 28 58 34      0      0 0.6930406201
## D8M339    8       2 28 55 35      0      0 0.5033645208
## D8M178    8       5 25 58 32      0      0 0.6502263262
## D8M242    8      30 23 42 25      0      0 0.7831394949
## D8M213    8       3 30 52 35      0      0 0.3922337029
## D8M156    8      18 25 49 28      0      0 0.8464817249
## D9M247    9       1 30 59 30      0      0 0.9958071340
## D9M328    9      10 30 49 31      0      0 0.5149752882
## D9M106    9       0 27 58 35      0      0 0.5488116361
## D9M269    9      38 22 41 19      0      0 0.8960526581
## D9M346    9      22 23 43 32      0      0 0.2098789192
## D9M55     9      30 27 42 21      0      0 0.5488116361
## D9M18     9      30 28 41 21      0      0 0.4065696597
## D10M298  10       1 40 59 20      0      0 0.0345431429
## D10M294  10      30 31 45 14      0      0 0.0403117975
## D10M42_  10       4 30 63 23      0      0 0.4259436255
## D10M10   10      11 28 61 20      0      0 0.2560492882
## D10M233  10       0 29 62 29      0      0 0.9355069850
## D11M78   11      30 23 47 20      0      0 0.8278784881
## D11M20   11       3 23 63 31      0      0 0.4093591491
## D11M242  11       9 22 57 32      0      0 0.3900651738
## D11M356  11      31 18 43 28      0      0 0.3090793301
## D11M327  11       0 28 54 38      0      0 0.2385125539
## D11M333  11       3 27 57 33      0      0 0.7074036474
## D12M105  12      30 31 35 24      0      0 0.0628712266
## D12M46   12      33 34 33 20      0      0 0.0083344619
## D12M34   12       2 37 62 19      0      0 0.0551165588
## D12M5    12       6 38 51 25      0      0 0.1207497463
## D12M99   12      30 29 43 18      0      0 0.2385125539
## D12M150  12       5 33 50 32      0      0 0.3727092993
## D13M59   13       0 34 14  7      0     65 0.7007839983
## D13M88   13       6 33 61 20      0      0 0.1715024212
## D13M21   13      17 35 47 21      0      0 0.1006489555
## D13M39   13      32 33 35 20      0      0 0.0232520115
## D13M167  13      31 32 35 22      0      0 0.0427799650
## D13M99   13       0 48 49 23      0      0 0.0007281525
## D13M233  13      14 41 43 22      0      0 0.0050294088
## D13M106  13       0 45 55 20      0      0 0.0036065631
## D13M147  13       0 45 55 20      0      0 0.0036065631
## D13M226  13      28 27 49 16      0      0 0.2207179658
## D13M290  13       1 42 58 19      0      0 0.0112972801
## D13M151  13       1 38 58 23      0      0 0.1453557012
## D14M14   14      32 20 44 24      0      0 0.8337529181
## D14M115  14       9 27 57 27      0      0 0.9602702338
## D14M265  14       4 30 61 25      0      0 0.6902581264
## D14M266  14      30 19 53 18      0      0 0.2385125539
## D15M226  15       8 27 58 27      0      0 0.9310627797
## D15M100  15      18 27 47 28      0      0 0.7235906755
## D15M209  15       1 31 55 33      0      0 0.6880116006
## D15M144  15      30 24 41 25      0      0 0.6930406201
## D15M68   15       3 33 51 33      0      0 0.3823042729
## D15M239  15       0 32 57 31      0      0 0.8535652128
## D15M241  15       5 35 54 26      0      0 0.3995600132
## D15M34   15      42 23 31 24      0      0 0.1913126738
## D16M154  16      30 28 35 27      0      0 0.1071705988
## D16M4    16       1 33 53 33      0      0 0.4916028846
## D16M139  16      31 26 39 24      0      0 0.4844606343
## D16M86   16      30 24 45 21      0      0 0.9048374180
## D17M260  17      30 22 46 22      0      0 0.9780228725
## D17M66   17      30 19 47 24      0      0 0.6930406201
## D17M88   17      15 22 57 26      0      0 0.5838593069
## D17M129  17       3 27 58 32      0      0 0.8041666799
## D18M94   18      30 30 40 20      0      0 0.1888756028
## D18M58   18       7 32 54 27      0      0 0.7175889212
## D18M106  18      40 23 40 17      0      0 0.6376281516
## D18M186  18       0 35 53 32      0      0 0.4099718965
## D19M68   19      30 23 45 22      0      0 0.9889503893
## D19M117  19      34 26 43 17      0      0 0.3899017615
## D19M65   19      32 27 43 18      0      0 0.3893869039
## D19M10   19      31 26  0  0      0     63 0.3586266707
## DXM186    X       1 69 50  0      0      0 0.0815562004
## DXM64     X       5 52 63  0      0      0 0.3050069459
\end{verbatim}

\section{Calculate Genotype
Probabilites}\label{calculate-genotype-probabilites}

Before performing interval mapping or generating effect plots, R/qtl
requires genotype probabilities at positions between observed markers.
The \texttt{calc.genoprob()} function computes these probabilities along
each chromosome, accounting for recombination rates and a small
genotyping error probability (\texttt{error.prob=0.01}). Setting
\texttt{step=1} calculates genotype probabilities at 1cM intervals,
producing a smoother and more accurate representation of the underlying
genetic information needed for QTL detection.

\begin{Shaded}
\begin{Highlighting}[]
\CommentTok{\# Needed for effect plots and interval mapping}
\NormalTok{cross }\OtherTok{\textless{}{-}} \FunctionTok{calc.genoprob}\NormalTok{(cross, }\AttributeTok{step=}\DecValTok{1}\NormalTok{, }\AttributeTok{error.prob=}\FloatTok{0.01}\NormalTok{)}
\end{Highlighting}
\end{Shaded}

\section{Perform Genome Scan}\label{perform-genome-scan}

To identify genomic regions associated with the phenotype, a genome-wide
QTL scan is performed using \texttt{scanone()}. Here, the analysis uses
Haley--Knott regression (\texttt{method="hk"}) (Haley \& Knott, 1992), a
fast and robust approximation for interval mapping. The function tests
each genomic position for association with the phenotype \texttt{"T264"}
and returns a LOD (Logarithm of the Odds) score profile across all
chromosomes. The resulting plot visualizes these LOD scores, allowing
identification of peaks that may represent putative QTL.

\begin{Shaded}
\begin{Highlighting}[]
\CommentTok{\# Haley{-}Knott regression genome scan}
\NormalTok{scan1 }\OtherTok{\textless{}{-}} \FunctionTok{scanone}\NormalTok{(cross, }\AttributeTok{pheno.col=}\StringTok{"T264"}\NormalTok{, }\AttributeTok{method=}\StringTok{"hk"}\NormalTok{)  }
\end{Highlighting}
\end{Shaded}

\begin{verbatim}
## Warning in checkcovar(cross, pheno.col, addcovar, intcovar, perm.strata, : Dropping 4 individuals with missing phenotypes.
\end{verbatim}

\begin{Shaded}
\begin{Highlighting}[]
\CommentTok{\# Enhanced plot with title and axis labels}
\FunctionTok{plot}\NormalTok{(scan1, }
     \AttributeTok{main =} \StringTok{"Genome{-}Wide QTL Scan for Trait T264"}\NormalTok{,}
     \AttributeTok{xlab =} \StringTok{"Chromosome"}\NormalTok{, }
     \AttributeTok{ylab =} \StringTok{"LOD Score"}\NormalTok{,}
     \AttributeTok{col =} \StringTok{"black"}\NormalTok{, }
     \AttributeTok{lwd =} \DecValTok{2}\NormalTok{)}
\end{Highlighting}
\end{Shaded}

\pandocbounded{\includegraphics[keepaspectratio]{Rqtl_Demo_files/figure-latex/unnamed-chunk-5-1.pdf}}

\section{Permutation Test for
Signifcance}\label{permutation-test-for-signifcance}

To determine whether the observed LOD scores exceed those expected by
chance, a permutation test is performed. The function \texttt{scanone()}
is run 1,000 times with shuffled phenotype labels
(\texttt{n.perm=1000}), generating an empirical null distribution of LOD
scores. This approach provides a robust, data-driven significance
threshold that accounts for genome-wide multiple testing. The
\texttt{summary()} function then compares the observed scan results to
the permutation-derived thresholds at the \(\alpha\) = 0.05 level,
identifying which peaks represent statistically significant QTL.

\begin{Shaded}
\begin{Highlighting}[]
\NormalTok{perm }\OtherTok{\textless{}{-}} \FunctionTok{scanone}\NormalTok{(cross, }\AttributeTok{pheno.col=}\StringTok{"T264"}\NormalTok{, }\AttributeTok{method=}\StringTok{"hk"}\NormalTok{, }\AttributeTok{n.perm=}\DecValTok{1000}\NormalTok{)}
\end{Highlighting}
\end{Shaded}

\begin{verbatim}
## Warning in checkcovar(cross, pheno.col, addcovar, intcovar, perm.strata, : Dropping 4 individuals with missing phenotypes.
\end{verbatim}

\begin{verbatim}
## Doing permutation in batch mode ...
\end{verbatim}

\begin{Shaded}
\begin{Highlighting}[]
\FunctionTok{summary}\NormalTok{(scan1, }\AttributeTok{perms=}\NormalTok{perm, }\AttributeTok{alpha=}\FloatTok{0.05}\NormalTok{)}
\end{Highlighting}
\end{Shaded}

\begin{verbatim}
##          chr  pos  lod
## c5.loc28   5 28.0 6.68
## D13M147   13 26.2 5.92
\end{verbatim}

\section{Plotting Significant QTLs}\label{plotting-significant-qtls}

This code visualizes the genome scan results and highlights
statistically significant QTLs. First, the 5\% genome-wide significance
threshold is extracted from the permutation test using
\texttt{summary(perm,\ alpha=0.05)}. The LOD score profile from the
genome scan is plotted with \texttt{plot(scan1)}, and a horizontal
dashed red line (\texttt{abline()}) is added at the threshold value.
Peaks that rise above this line represent loci that are significant at
the genome-wide level.

\begin{Shaded}
\begin{Highlighting}[]
\CommentTok{\# Get the 5\% genome{-}wide significance threshold}
\NormalTok{lod\_thresh }\OtherTok{\textless{}{-}} \FunctionTok{summary}\NormalTok{(perm, }\AttributeTok{alpha=}\FloatTok{0.05}\NormalTok{)}

\CommentTok{\# Plot LOD profile with improved labels}
\FunctionTok{plot}\NormalTok{(scan1, }
     \AttributeTok{main =} \StringTok{"Genome{-}Wide LOD Profile with 5\% Significance Threshold"}\NormalTok{,}
     \AttributeTok{xlab =} \StringTok{"Chromosome"}\NormalTok{, }
     \AttributeTok{ylab =} \StringTok{"LOD Score"}\NormalTok{,}
     \AttributeTok{col =} \StringTok{"black"}\NormalTok{, }
     \AttributeTok{lwd =} \DecValTok{2}\NormalTok{)}

\CommentTok{\# Add horizontal line for genome{-}wide significance}
\FunctionTok{abline}\NormalTok{(}\AttributeTok{h =}\NormalTok{ lod\_thresh, }\AttributeTok{col =} \StringTok{"red"}\NormalTok{, }\AttributeTok{lty =} \DecValTok{2}\NormalTok{)}
\end{Highlighting}
\end{Shaded}

\pandocbounded{\includegraphics[keepaspectratio]{Rqtl_Demo_files/figure-latex/unnamed-chunk-7-1.pdf}}

\section{Identify QTL Confidence
Interval}\label{identify-qtl-confidence-interval}

This code identifies the approximate confidence intervals for detected
QTLs on specific chromosomes. The \texttt{lodint()} function calculates
the region around the peak LOD score where the LOD drops by a specified
value (\texttt{drop=1.5} corresponds roughly to a 95\% confidence
interval). In this example, confidence intervals are computed for
chromosomes 5 and 13. The resulting intervals indicate the genomic
regions most likely to contain the causal loci underlying the trait of
interest.

\begin{Shaded}
\begin{Highlighting}[]
\CommentTok{\# LOD interval (95\% confidence) for a chromosome}
\NormalTok{ci5 }\OtherTok{\textless{}{-}} \FunctionTok{lodint}\NormalTok{(scan1, }\AttributeTok{chr=}\StringTok{"5"}\NormalTok{, }\AttributeTok{drop=}\FloatTok{1.5}\NormalTok{)}
\NormalTok{ci5}
\end{Highlighting}
\end{Shaded}

\begin{verbatim}
##          chr pos      lod
## c5.loc15   5  15 5.054036
## c5.loc28   5  28 6.679752
## c5.loc38   5  38 4.842088
\end{verbatim}

\begin{Shaded}
\begin{Highlighting}[]
\NormalTok{ci13 }\OtherTok{\textless{}{-}} \FunctionTok{lodint}\NormalTok{(scan1, }\AttributeTok{chr=}\StringTok{"13"}\NormalTok{, }\AttributeTok{drop=}\FloatTok{1.5}\NormalTok{)}
\NormalTok{ci13}
\end{Highlighting}
\end{Shaded}

\begin{verbatim}
##           chr      pos      lod
## c13.loc17  13 17.00000 4.387866
## D13M147    13 26.15954 5.923396
## c13.loc30  13 30.00000 4.152743
\end{verbatim}

\section{Plot Confidence Intervals}\label{plot-confidence-intervals}

This section visualizes the LOD profiles for chromosomes 5 and 13 from
the genome-wide QTL scan. Horizontal red dashed lines represent the 5\%
genome-wide significance threshold, indicating which peaks are
statistically significant. Blue vertical dashed lines denote the 95\%
confidence intervals around detected QTLs, providing a range where the
true QTL is likely located. Red points mark the peak markers within
these intervals (\texttt{c5.loc28} for chromosome 5 and \texttt{D13M147}
for chromosome 13), highlighting the loci with the strongest association
to the trait of interest. This visualization helps to quickly identify
significant QTLs and their approximate genomic locations.

\begin{Shaded}
\begin{Highlighting}[]
\CommentTok{\# LOD profile plot for chromosome 5}
\FunctionTok{plot}\NormalTok{(scan1, }\AttributeTok{chr=}\DecValTok{5}\NormalTok{, }\AttributeTok{main=}\StringTok{"LOD Profile for Chromosome 5"}\NormalTok{, }\AttributeTok{ylab=}\StringTok{"LOD score"}\NormalTok{)}

\CommentTok{\# Add horizontal significance threshold}
\FunctionTok{abline}\NormalTok{(}\AttributeTok{h=}\NormalTok{lod\_thresh, }\AttributeTok{col=}\StringTok{"red"}\NormalTok{, }\AttributeTok{lty=}\DecValTok{2}\NormalTok{)}

\CommentTok{\# Add vertical lines for CI}
\FunctionTok{abline}\NormalTok{(}\AttributeTok{v=}\DecValTok{15}\NormalTok{, }\AttributeTok{col=}\StringTok{"blue"}\NormalTok{, }\AttributeTok{lty=}\DecValTok{2}\NormalTok{) }\CommentTok{\# start of CI}
\FunctionTok{abline}\NormalTok{(}\AttributeTok{v=}\DecValTok{38}\NormalTok{, }\AttributeTok{col=}\StringTok{"blue"}\NormalTok{, }\AttributeTok{lty=}\DecValTok{2}\NormalTok{) }\CommentTok{\# end of CI}

\CommentTok{\# Peak marker}
\FunctionTok{points}\NormalTok{(}\AttributeTok{x=}\DecValTok{28}\NormalTok{, }\AttributeTok{y=}\FloatTok{6.68}\NormalTok{, }\AttributeTok{col=}\StringTok{"red"}\NormalTok{, }\AttributeTok{pch=}\DecValTok{19}\NormalTok{) }\CommentTok{\# c5.loc28}
\end{Highlighting}
\end{Shaded}

\pandocbounded{\includegraphics[keepaspectratio]{Rqtl_Demo_files/figure-latex/unnamed-chunk-9-1.pdf}}

\begin{Shaded}
\begin{Highlighting}[]
\CommentTok{\# LOD profile plot for chromosome 13}
\FunctionTok{plot}\NormalTok{(scan1, }\AttributeTok{chr=}\DecValTok{13}\NormalTok{, }\AttributeTok{main=}\StringTok{"LOD Profile for Chromosome 13"}\NormalTok{, }\AttributeTok{ylab=}\StringTok{"LOD score"}\NormalTok{)}

\CommentTok{\# Add horizontal significance threshold}
\FunctionTok{abline}\NormalTok{(}\AttributeTok{h=}\NormalTok{lod\_thresh, }\AttributeTok{col=}\StringTok{"red"}\NormalTok{, }\AttributeTok{lty=}\DecValTok{2}\NormalTok{)}

\CommentTok{\# Add vertical lines for CI}
\FunctionTok{abline}\NormalTok{(}\AttributeTok{v=}\DecValTok{17}\NormalTok{, }\AttributeTok{col=}\StringTok{"blue"}\NormalTok{, }\AttributeTok{lty=}\DecValTok{2}\NormalTok{) }\CommentTok{\# start of CI}
\FunctionTok{abline}\NormalTok{(}\AttributeTok{v=}\DecValTok{30}\NormalTok{, }\AttributeTok{col=}\StringTok{"blue"}\NormalTok{, }\AttributeTok{lty=}\DecValTok{2}\NormalTok{) }\CommentTok{\# end of CI}

\CommentTok{\# Peak marker}
\FunctionTok{points}\NormalTok{(}\AttributeTok{x=}\FloatTok{26.16}\NormalTok{, }\AttributeTok{y=}\FloatTok{5.92}\NormalTok{, }\AttributeTok{col=}\StringTok{"red"}\NormalTok{, }\AttributeTok{pch=}\DecValTok{19}\NormalTok{) }\CommentTok{\# D13M147}
\end{Highlighting}
\end{Shaded}

\pandocbounded{\includegraphics[keepaspectratio]{Rqtl_Demo_files/figure-latex/unnamed-chunk-9-2.pdf}}

\section{Plot Marker Effects}\label{plot-marker-effects}

This code visualizes the effect of a specific marker on the phenotype.
The \texttt{effectplot()} function shows how different genotypes at the
chosen marker (here, \texttt{D13M147}) influence the trait values. This
allows you to assess the magnitude and direction of the genetic effect
at that locus. Note that if genotype probabilities were not previously
calculated with \texttt{calc.genoprob()}, the function will
automatically perform that step before plotting.

\begin{Shaded}
\begin{Highlighting}[]
\CommentTok{\# Plot effect at marker D13M147 with enhanced visualization}
\FunctionTok{effectplot}\NormalTok{(cross, }\AttributeTok{mname1 =} \StringTok{"D13M147"}\NormalTok{,}
           \AttributeTok{main =} \StringTok{"Genotypic Effect at Marker D13M147"}\NormalTok{,}
           \AttributeTok{xlab =} \StringTok{"Genotype"}\NormalTok{,}
           \AttributeTok{ylab =} \StringTok{"Trait Value (T264)"}\NormalTok{)}
\end{Highlighting}
\end{Shaded}

\begin{verbatim}
## Warning in effectplot(cross, mname1 = "D13M147", main = "Genotypic Effect at
## Marker D13M147", : -Running sim.geno.
\end{verbatim}

\pandocbounded{\includegraphics[keepaspectratio]{Rqtl_Demo_files/figure-latex/unnamed-chunk-10-1.pdf}}

\section{Conclusion}\label{conclusion}

This workflow demonstrates a complete pipeline for QTL mapping using
R/qtl, from importing genotype and phenotype data to performing genome
scans, permutation testing, identifying significant QTLs, and estimating
confidence intervals and effect sizes. Using the \texttt{listeria} F2
intercross dataset as a proof-of-concept, all steps including data
quality checks, genotype probability calculation, and effect plotting
were successfully executed, confirming that the pipeline functions as
intended. Once genotype and phenotype data from the F2 Poecilia
population are available, this workflow can be directly applied,
enabling identification of loci associated with dorsal fin traits and
facilitating downstream analyses to investigate the genetic architecture
of sexually selected traits. The modular structure of this pipeline
ensures reproducibility and allows easy adaptation to new datasets,
making it a reliable tool for quantitative genetics studies.

\section{References}\label{references}

Broman, K. W., Wu, H., Sen, Ś., \& Churchill, G. A. (2003). R/qtl: QTL
mapping in experimental crosses. Bioinformatics, 19(7), 889--890.
\url{https://doi.org/10.1093/bioinformatics/btg112}

Haley, C. S., \& Knott, S. A. (1992). A simple regression method for
mapping quantitative trait loci in line crosses using flanking markers.
Heredity, 69(4), 315--324. \url{https://doi.org/10.1038/hdy.1992.131}

\end{document}
